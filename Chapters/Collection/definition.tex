%!TEX root = ../../main.tex

\section{Defining an Event}
\label{DefiningEvent}
In Section \ref{background:event} we discussed that, although there is a consensus that events have a temporal nature, there is a lack of a concrete definition of event for the evaluation of event detection approaches on Twitter.
This makes it difficult to compare the effectiveness of different event detection approaches as different definitions may lead to different judgements being made for the same topic.

To solve these issues, we take the most basic definition of event (something that happens at a particular time and place), and introduce the requirement that an event should be \textit{significant}.
By requiring that an event is significant, we are able to filter out every-day personal and trivial events which are extremely common on Twitter (such as getting a phone call or going to the gym).
\begin{description}
\item[Event:] An event is a \emph{significant} thing that happens at some specific time and place.
\end{description}
As discussed in Chapter \ref{background:event}, it was reasonable to assume that all events in the TDT datasets were significant events due to the use of newswire documents, something which is not true in the case of Twitter.
Given this, we attempt to model our definition of significance so that an event under our definition would be of a similar level of significance to those found in the TDT datasets, despite the disparity between the 2 sources.
\begin{description}
\item[Significant:] Something is significant if it may be discussed in the media. For example, you may read a news article or watch a news report about it.
\end{description}
It is important to note that something does not necessarily have to be discussed by the media in order for it to be an event, we simply use this as an indication of the level of significance required for something to be considered an event.
Whilst this is still somewhat subjective, we believe that it is impossible to further restrict significance whilst keeping our definition of event generalizable.
Given this definition of event, our aim is to create a collection of significant events which have been discussed on Twitter, and set of relevance judgements for tweets which discuss the events.

We also note that by using the term \emph{significant} to qualify a topic as an event (or not), we allow for the definition of event to be kept constant even for different use-cases.
While we align our definition of \emph{significant} such that it brings our definition of event in line with that of the TDT project, we do not exclude the possibility that \emph{significant} could be altered to suit other use-cases.
For example, a system designed to assist the emergency services may only consider an event to be significant if it requires an emergency response, whereas a system designed for financial traders may define significant as something that might affect the price of a security.