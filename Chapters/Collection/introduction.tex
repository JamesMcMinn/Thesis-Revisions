%!TEX root = ../../main.tex
The lack of a large-scale Twitter corpus makes the evaluation and comparison of event detection approaches incredibly difficult.
Indeed, it is not uncommon for researchers to create a bespoke dataset and carry out a time-consuming and potentially biased manual evaluation of their event detection approach, for which the dataset and judgements are often not made publicly available.

A small number of Twitter corpora with relevance judgements have been made available (these are detailed in Section \ref{background:sec:collections}), however none are suitable for the large-scale evaluation of event detection approaches due to their small size or number of topics.
This is partially due to the massive scale of Twitter, which makes the creation of corpora difficult, time-consuming and expensive, but also due to disagreement on the definition of event, which can often make comparisons difficult or impossible.
Furthermore, Twitter's Terms of Service restrict the distribution of tweets, and do not allow the content of tweets to be distributed as part of a corpus.
Instead, tweet IDs must be released, and those wishing to use a dataset must register for a Twitter developer account and use Twitter's rate-limited API to crawl the tweets.
As a result, there are very few publicly available Twitter corpora, and in some cases, corpora from other medium are used in place of a Twitter corpus \citep{Aggarwal12,Petrovic:2010:SFS:1857999.1858020,Petrovic:2012:UPI:2382029.2382072}.
However, it is not clear that effectiveness of event detection approaches on a non-Twitter corpus is comparable to effectiveness on a Twitter corpus, with some evidence suggesting that this is not the case \citep{Petrovic:2012:UPI:2382029.2382072}.
This scenario leads to a situation in which event detection techniques are not properly benchmarked.

As we discussed in Section \ref{background:sec:collections}, there are a number of issues that make it impossible to use a standard pooling approach to create a collection for the evaluation of event detection approaches on Twitter.
This motivates the need for a new methodology to create large-scale test collections for the evaluation of event detection approaches.

In this chapter, we propose a new, Twitter centric, definition of event which we believe better fits how users perceive events and how events are discussed on Twitter.
Then, using this new definition, we propose a new methodology for creating a set of relevance judgements for the evaluation of event detection approaches on Twitter.

To do so, we create a collection of 120 million tweets, and by implement two existing event detection approaches, we extract a set of candidate events and their associated tweets.
We use crowdsourcing to evaluate each candidate event using our definition of event, and gather tweet-level relevance judgements for each event.
We also extract a list of significant events from Wikipedia's Current Events Portal, and for each event, extract potentially relevant tweets and use crowdsourcing to gather relevance judgements.
We then use a clustering approach to merge events from each approach so that each event discusses the same real-world event.
We make this corpus available for other researchers to use at \texttt{http://mir.dcs.gla.ac.uk/resources/}.

% To do so, we use two existing event detection approaches and the Wikipedia Current Events Portal to generate a set of candidate events and associated tweets.
% We then use crowdsourcing to evaluate if a candidate event fits our definition of event, and create relevance judgements for each of the events.
% We then merge events from each of the sources which discuss the same real-world event, and extending our original work \citep{McMinn2013}, examine the quality of the clustering.
% Finally, we make the corpus and judgements available for research and further development.

This chapter has a number of novel contributions:
\begin{itemize}
\item We create the first test collection of this scale for event detection on Twitter
\item We propose a new and robust definition of `event' that deals with the nuances of events on Twitter
\item We propose a novel methodology for the creation of relevance judgements using two state-of-the-art event detection approaches, Wikipedia, and Amazon Mechanical Turk
\item We study the quality and characteristics of the corpus and the relevance judgements
\end{itemize}
