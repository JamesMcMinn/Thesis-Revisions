%!TEX root = ../../main.tex

\section{Background}
The automatic classification of story newsworthiness is not new, and is a common pre-processing step in credibility assessment and summarization \citep{Noyunsan17, Madhawa15}.

\cite{Noyunsan16} used cosine similarity to filter out non-newsworthy posts on Facebook by comparing new posts to known non-newsworthy posts.
They created a groundtruth of 1,005 non-newsworthy Facebook posts, annotated and collected via a browser extension installed by volunteers.
Their approach compares each new post to every post in the groundtruth, and classifies posts as non-newsworthy (noise) if its similarity to any document in the groundtruth is above a threshold.
They later examined how the removal of non-newsworthy posts affects credibility assessment and found that by removing non-newsworthy posts they could increase the credibility assessment effectiveness \citep{Noyunsan17}.

\cite{Madhawa15} examined the performance of difference classifiers for labelling tweets as newsworthy or noise as the first step in a pipeline for the summarization of events on Twitter.
However, \cite{Madhawa15} define newsworthy and noise as `objectivity' and `subjectivity' respectively.
Although this distinction is true in many cases, and provides a useful distinction for summarization, it does not apply in all cases.
Subjective comments by noteworthy individuals, such as politicians or public figures, are often a newsworthy event.
For this reason, we argue that simple objectivity and subjectivity labelling does not capture the true newsworthiness of a post, and instead a more robust newsworthiness \emph{score} is required.
