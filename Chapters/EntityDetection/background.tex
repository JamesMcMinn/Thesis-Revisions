%!TEX root = ../../main.tex

% \section{Background}
% \todo{In this section, we present a brief survey of work related to the detection and analysis of events. We give a brief overview of the Topic Detection and Tracking project and describe the role of social media is breaking news and events. We then discuss a number of current event detection approaches for Twitter, and motivate the need for a new approach. Finally, we describe the role of named entities in events, and discuss why we believe this can be used to improve event detection in social media.}

% \todo{
% Although some these approaches have been evaluated, none of them have been evaluated in-depth on a large-scale, publicity available, test collection for Twitter. This makes comparing different event detection approaches very difficult. In this paper, we hope to provide the first in-depth evaluation of an event detection approach on Twitter using a publicity available test collection.}

\section{Named Entities in Events}
\label{detection:sec:entityEvents}
We believe that named entities play a key role in describing events, such as the people involved, or the location where the event took place. Without this information, or some other contextual clue, it is unreasonable to expect a person or machine to determine the specifics of an event. For example, given the document ``\textit{A bomb exploded.}'', it is impossible to determine who was involved or where the event took place -- we are only able to say that a bomb exploded somewhere (assuming the tweet itself is true). Only by introducing entities or other contextual information can we begin to determine the specifics of an event: ``\textit{\textbf{Boko Haram} claims responsibility for a bomb which exploded in the north-east \textbf{Nigerian} town of \textbf{Potiskum}.}''. Given this information, we are now able to say who was involved (Boko Haram), where the event took place (Potiskum, Nigeria), and due to Twitter real-time nature, infer with some confidence that the event took place recently.

Clearly entities play an important role in events, and this can be exploited in a number of ways.  It is not unreasonable to assume that tweets which do not discuss the same entities (that is to say, do not share at least one common entity) are unlikely to discuss the same event. Conversely, it is not unreasonable to assume that tweets which discuss the same entities are more likely to discuss the same event. We believe that by exploiting the role that entities play in real-world events, we can produce an effective and efficient event detection algorithm. Furthermore, by using the relationship between entities within events, we believe that we can improve upon current state-of-the-art performance using semantic links between entities to improve event-based retrieval.

The use of named entities for event detection has been suggest in the past,  \cite{Kumaran:2004:TCN:1008992.1009044} used named entities to improve First Story Detection performance on Topic Detection and Tracking corpora, however there was concern that its effectiveness would be too low when dealing with the low quality and noisy content found on Twitter. \cite{DBLP:conf/sigir/LiWHYDSL12} examined the performance of current state-of-the-art Named Entity Recognition (NER) software on Twitter, and although they found that it was possible to significantly improve upon current performance, their results show that out-of-the-box solutions, such as the Stanford NER\footnote{http://www-nlp.stanford.edu/software/tagger.shtml}, perform adequately for most tasks.

\cite{Choudhury11extractingsemantic} examined how linguistic features and background knowledge could be used to detect specific types of sports events with high precision. However their approach requires significant domain knowledge of the sporting events and large amounts of manual labeling to prepare a classifier for each specific type of event, making it difficult and time-consuming to generalize.

\cite{Ritter:2012:ODE:2339530.2339704} used named entities, ``event phrases'' and temporal expressions to extract a calendar of significant events from Twitter.
However, their approach requires that tweets contain temporal resolution phrases, such as ``tomorrow'' or ``Wednesday'' for their approach to resolve between an entity and a specific time.
This means that smaller events, which do no generate much discussion before they happen, are unlikely to be detected. Additionally unexpected events, which are often the events which are of most interest, are unlikely to be detected as they will be discussed as they happen, and without any temporal resolution phrases.

Most similar to our work is that of \cite{Reuters2017}, who use a similar partitioning technique to efficiently cluster documents by splitting the tweet term space into groups they call `semantic categories', such as named entities, mentions, hashtags, nouns and verbs.
To ensure that only novel events are reported, they compare each new event to old events, filtering out any events that have already been reported.
They merge events by comparing the top 20\% of entity terms, based on term frequency, similar to how we merge events based on entity co-occurrence.
They evaluate their approach using the Events 2012 corpus from chapter \ref{chapter:collection}, finding that their approach outperforms the LSH approach when measured using NMI and B-Cubed cluster performance metrics.
It is important to note that the work presented in this chapter was first published in \cite{McMinn15}, 2 years prior to \cite{Reuters2017}.