%!TEX root = ../../main.tex

\newpage
\chapter{Background}
\label{chapter:background}

In this chapter, we review the background information required to understand the remainder of this thesis.
We draw on several areas of research, such as Information Retrieval (IR) and Natural Language Processing (NLP), and describe how documents are represented and compared, and detail a number of IR measures.
We then give a brief overview of the Topic Detection and Tracking project, and how it related to modern approaches for event detection on Twitter.
Finally, we give a survey of related work in the area of event detection Twitter and describe how it relates to the work presented in this thesis.


\section{Information Retrieval}
Fist we discuss the area of Information Retrieval, and describe some of the most common concepts that we use throughout this thesis.
In a traditional IR task, the goal is to take a query describing some information need, and return a ranked list of documents that are relevant to that query.
In the remainder of this section, we examine some of the basic techniques used to complete this task, and describe how they relate to event detection on Twitter.

\subsection{Document Representation}

One of the most commonly used document and query representations is the Vector Space model \citep{salton1975vector}. The Vector Space model represents both queries and documents as a vector, where each term in the document corresponds to a dimension in the vector.
Terms that occur in a document will have a non-zero value in the vector, while terms that do not appear will have a value of 0.
The Vector Space model is used throughout this thesis to represent documents (tweets).

\subsubsection{TF-IDF}
One of the most common ways of calculating the weight a term has in a vector is known as the Term Frequency-Inverse Document Frequency model, or TF-IDF.
TF-IDF attempts to estimate the importance of a term with regards to both the collection as a whole and document itself.

The TF component is concerned with the weight of the term in relation to the document, the basic premise being that the more frequently a term occurs in a document, the more weight it should carry.
The IDF component, on the other hand, is concerned with the weight of the term in relation to the corpus as a whole.
The IDF component attempts to estimate the amount of information that a term carries on the basis that rare terms should be given more weight as their presence in a document is more likely to be an indication of topic, whilst common terms should be given less weight as they are less topically specific.
TF-IDF is calculated as so:
\begin{displaymath}
	w_{t,d} = \mathrm{tf}_{t,d} \cdot \log{\frac{|D|}{|\{d' \in D \, | \, t \in d'\}|}}
\end{displaymath}
where \(\mathrm{tf}_{t,d}\) is the term frequency of term \(t\) in document \(d\), \(|D|\) is the total number of documents in the collection, and \(|\{d' \in D \, | \, t \in d'\}|\) is the number of documents contain term \(t\).
Note that for retrieval tasks on Twitter, the TF component is often ignored as tweets are very short, and generally do not contain repeated terms.

\subsubsection{Cosine}

Many IR tasks, including event detection, are concerned with the the notion of similarity.
One such similarity measure is cosine similarity, which measures the cosine of the angle between two documents in vector space:
\begin{displaymath}
	\cos{\theta} = \frac{\mathbf{d_1} \cdot \mathbf{d_2}}{\left\| \mathbf{d_1} \right\| \left \| \mathbf{d_2} \right\|}
\end{displaymath}
where \(\mathbf{d_1} \cdot \mathbf{d_2}\) is the dot product of term weighted vectors for the documents, and the norm for vector \(\mathbf{d}\) is calculated as such:
\begin{displaymath}
	\left\| \mathbf{d} \right\| = \sqrt{\sum_{i=1}^n d_i^2}
\end{displaymath}

\subsection{Evaluation Measures}
Throughout this thesis, we use a number of IR standard evaluation metrics, commonly used in IR evaluations.
These measures as based on the concept of relevance.
In IR, relevance is determined based on some information need, usually represented as a textual query.
In event detection, there is no query.
Instead, relevance is defined in relation to an event, either by subjectively evaluating if a cluster of tweets meets some definition of `event', or by matching tweets in the cluster to a pre-determined event in the relevance judgements (represented by a set of tweets).
We discuss this further is chapter \ref{chapter:detection}.

\subsubsection{Recall}

Recall is measured as the fraction of relevant documents that at retrieved from all possible relevant documents. Recall is give as:

\begin{displaymath}
	\text{recall}=\frac{|\{\text{relevant documents}\}\cap\{\text{retrieved documents}\}|}{|\{\text{relevant documents}\}|}
\end{displaymath}

\subsubsection{Precision}
In a traditional IR setting, precision is measured as the fraction of retrieved documents that are relevant to a given query.
For the purpose of evaluating event detection approaches, the lack of a query means that we must consider relevance differently; either in terms of meeting some definition of `event', or by matching back to some event from the relevance judgements.
How this is done is examined in chapters \ref{chapter:collection} and \ref{chapter:detection}.

In a standard IR setting, precision is given as:
\begin{displaymath}
	\text{precision}=\frac{|\{\text{relevant documents}\}\cap\{\text{retrieved documents}\}|}{|\{\text{retrieved documents}\}|}
\end{displaymath}

\subsubsection{F-measure}
Individually, precision and recall show only one aspect of performance.
Perfect recall can be achieved by returning every document in the collection,
whilst perfect precision can be achieved by returning none of them.

F-measure takes both precision and recall into consideration, and combines them into a single score. F-measure ranges between 1 at best (a perfect systems) and 0 at worst. Although it is possible to weight F-measure to prefer precision or recall, the most commonly used F-measure gives equal weight to both, and is the harmonic mean of precision and recall. This is usually called F1 score, and is used throughout this thesis:

\begin{displaymath}
	F_1 = 2 \cdot \frac{1}{\tfrac{1}{\mathrm{recall}} + \tfrac{1}{\mathrm{precision}}} = 2 \cdot \frac{\mathrm{precision} \cdot \mathrm{recall}}{\mathrm{precision} + \mathrm{recall}}
\end{displaymath}

\subsubsection{B Cubed}


\section{Topic Detection and Tracking}
The Topic Detection and Tracking (TDT) project began in 1997, and was a body of research and an evaluation paradigm that addressed the event-based organization of broadcast news \citep{Allan:2002:ITD:772260.772262}. The motivation behind the project was a system which was capable of monitoring broadcast news and could produce an alert when a new event occurred.

The project was split into five distinct subtasks \citep{Allan:2002:ITD:772260.772262}:

\begin{itemize}
\item \textbf{Story Segmentation} was the task of dividing audio transcripts taken from news shows into individual stories.

\item \textbf{First Story Detection} was the task of recognizing the onset of a new topic (event) in the stream of news stories.

\item \textbf{Cluster Detection} was the task of grouping all stories as they arrived, based on the topics they discuss.

\item \textbf{Tracking} was a task that required systems to identify news stories  discussing the same topic as a set of sample stories, by monitoring the stream of news stories as they appeared.

\item \textbf{Story Link Detection} was the task of deciding whether two randomly selected stories discussed the same topic (event).
\end{itemize}

With the exception of Story Segmentation, the tasks focus on various aspects of document clustering, with the main difference between each task being how it is evaluated.
Of the 5 tasks, First Story Detection (FSD, although often called New Event Detection, NED) was considered the most difficult \citep{Allan:2000:FSD:354756.354843} as an effective FSD system must be effective across all of the tasks to perform well.

Story Segmentation differs in nature from the others as it pertains specifically to a pre-processing step that must be performed before the other tasks can be carried out. All other tasks work with stories, however the aim of the Story Segmentation task is to detect boundaries between stories automatically using transcripts from the spoken audio of broadcast news shows. Of all the tasks of the TDT project, Story Segmentation is the least relevant to event detection on Twitter as tweets almost always discuss only a single topic due to their short length.

\subsection{Approaches to TDT}
\label{background:sec:tdt}
\begin{algorithm}
	\DontPrintSemicolon
	\KwIn{Minimum similarity threshold $m$}
	index $\gets []$ \\
	clusters $\gets$ [] \\

	\ForEach{document d in the stream} {
		$S(d) \gets \emptyset$ \tcp*{set of documents that share a term with $d$}

		\ForEach{term t in d}{
			\ForEach{document d' in index[t]} {
				$S(d) \gets S(d) \cup d'$ \\
			}
			index[$t$] $\gets$ index[$t$] $\cup$ $d$ \\
		}

		$c_{max} \gets 0$ \tcp*{maximum cosine between $d$ and documents in $S(d)$}
		$n_{d} \gets nil$ \tcp*{document with maximum cosine to $d$}
		\ForEach{document d' in S(d)}{
			$c$ := cosine($d$, $d'$) \\
			\If{$c > c_{max}$}{
				$c_{max} \gets c$ \\
				$n_d \gets d'$
			}
		}

		\eIf{$c_{max} \geq m$}{
			clusters[$d$] $\gets$ cluster[$n_d$]	\tcp*{Add to exisitng event}
		} {
			clusters[$d$] $\gets$ new cluster() \tcp*{Report a new event}
		}

	}
	\caption{A basic TDT approach, similar to that used by UMass and other TDT participants}
	\label{background:alg:tdt}
\end{algorithm}

At a high level, almost all of the approaches proposed by the TDT project follow the same nearest-neighbour clustering approach shown in algorithm \ref{background:alg:tdt}.
As documents appear in the stream, they are compared to every document that has been seen before (or, if an inverted index is used, than all documents they share a term with), and the similarity between the new document and every other document in the Corpus is calculated.
If one or more similar documents is found (based on some pre-defined threshold), add the new document to cluster containing the closest match, signaling that they discuss the same event.
If no similar documents are found, then a new event has been detected, and a new cluster is formed~\citep{Allan:2000:FSD:354756.354843}.

Unfortunately, the approach shown in algorithm \ref{background:alg:tdt} does not scale to extremely high volume streams.
The approach takes $O(|D_t|)$ time to compute in the worst case, and although the use of an inverted index helps to reduce the average case, it continues to take increasingly longer to process new documents as new data is seen.
This has obvious drawbacks when the volume of data is increased from a few thousand documents, to many hundreds of millions of documents per day, which quickly makes this approach computationally infeasible.

This basic model persisted throughout the TDT project, and variations of this model (with efficiency optimizations) are commonly found in event detection approaches for social media \citep{becker2011beyond,Petrovic:2010:SFS:1857999.1858020,Aggarwal12,conf/asunam/OzdikisSO12}.

\subsection{Performance of TDT Systems}
The TDT project produced some reasonably effective systems \citep{UMASS,Yang98,Allan:2005:TTD:1042435.1042899}, however performance was still far below that required for systems to be considered adequate for complete automation \citep{Allan:2000:FSD:354756.354843}.
When the TDT project ended in 2004, research in event detection slowed. It is believed that a plateau had been reached, and that without a radically different approach, the performance of these systems was unlikely to ever improve significantly \citep{Allan:2000:FSD:354756.354843}. This belief has remained true in the context of newswire documents, and many state-of-the-art systems are now over a decade old \citep{UMASS,Yang98,Allan:2005:TTD:1042435.1042899}.
The TDT project was a cornerstone in the development of event detection approaches and is responsible for much of the foundation on which many state-of-the-art event detection approaches for Twitter are based.

\section{Event Detection on Social Media}
In recent years, interest in event detection has been rekindled as social media data has become available for research. However social media posts introduce their own unique challenges which require novel approaches to overcome. The massive volume of data means that traditional event detection techniques are too slow for real-time applications. One of the requirements of an effective event detection system is the ability to monitor events in real-time. In addition, with high volume data streams, as is the case with Twitter, efficiency is crucial to an effective system.

Analysis of social media has received a lot of attention from the research community in recent years. However, much of this work has focused on blog and email streams  \citep{Zhao:2007:TIF:1619797.1619886,Jurgens:2009:EDB:1859650.1859652,Becker:2010:LSM:1718487.1718524,Nguyen:2011:ERR:2186701.2186707}, using datasets such as the Enron email dataset \cite{citeulike:7616867} or the BlogLines dataset \citep{Sia:2008:ECP:1401890.1401967}. More recently, focus has moved towards Twitter due to its popularity with individuals and organizations as a method of broadcast communication.

\cite{Hu:2012:BNT:2208636.2208672} demonstrated the effectiveness of Twitter as a medium for breaking news by examining how the news of Osama bin Laden's death broke on Twitter. They found that Twitter had broken the news, and as a result, millions already knew of his death before the official announcement.

\cite{Kwak:2010:TSN:1772690.1772751} analyzed the top trending topics to show that the majority of topics (over 85\%) are related to headline news or persistent news. They also found that once a tweet is retweeted, it can be expected to reach an average of 1,000 users.

\cite{WRN2012:osbornebieber} measured the delay between a new event appearing on Twitter, and the time taken for the same event to be updated on Wikipedia. Their results show that Twitter appears to be around 2 hours ahead of Wikipedia. They suggest that Wikipedia could be used as a filter, decreasing the number of spurious events, however at the cost of greatly increased latency. They demonstrate its effectiveness as a filter for their First Story Detection approach \citep{Petrovic:2010:SFS:1857999.1858020}, significantly decreasing the number of spurious events.

\section{Event Detection on Twitter}
A number of excellent survey papers covering Event Detection on Twitter have been published in recent years, including \cite{Hasan17}, \cite{Goswami2016}, and \cite{Atefeh2015}. Rather than replicate their work here, we invite interested readers to refer to these papers for a full survey all event detection approaches for Twitter.
Instead, we survey only the most novel and relevant work in this section.

\cite{sankaranarayanan2009twitterstand} proposed one of the first systems which aimed to detect breaking news and events from tweets. In conjunction with the Twitter garden-hose, they utilize a set of 2,000 handpicked \emph{seeds} -- twitter accounts who primarily post news-related tweets -- that they used as a trusted source of news.
An initial layer of filtering is performed on all incoming tweets, excluding those from the seeds, using naive Bayes classifier trained on a corpus of news and ``junk'' tweets. Next they use an online clustering approach that maintains a list of active clusters, along with an associated TF-IDF \citep{Salton:1988:TAA:54259.54260} feature-vector of the terms used by tweets in the cluster. An incoming tweet is added to a cluster if its similarity with the cluster's feature-vector is above a set threshold, measured using a time-decayed cosine similarity function. An inverted index is maintained so that only active clusters which contain at least some of the terms from the incoming tweet are used for comparison, and clusters with a time centroid greater than 3 days old are marked as inactive and removed from the index.
To further reduce the number of noisy clusters, \cite{sankaranarayanan2009twitterstand} impose an interesting restriction, in that for a cluster to remain active after \(k\) tweets have been added, one of the tweets must come from a seed account.
Despite a number of efficiency optimizations, their approach is unable to scale, or produce results in real-time. Unfortunately, no attempt is made to evaluate the effectiveness of their system other than a small number of empirical observations.

\cite{becker2011beyond} use the clustering approach proposed by \cite{Yang98} as part of the TDT project, and a filtering layer similar to \cite{sankaranarayanan2009twitterstand}.
However, filtering is performed after clustering has taken place, and they attempt to identify likely event clusters, rather than event tweets.
They use a number of features, such as top terms, and number of retweets, to classify clusters as either event or non-event.
Although evaluation of their approach seems to show that it is very effective, it is still based on a model designed for much lower volume streams, and is simply too slow to scale past very small corpora.

\subsection{Cluster Summerization Approach}
\cite{Aggarwal12} use a fixed number of clusters and cluster summaries to reduce the number of comparisons required for document clustering. Each cluster summary contains (i) a node-summary, which is a set of users and their frequencies and (ii) a content-summary, which is a set of words and their TF-IDF \cite{Salton:1988:TAA:54259.54260} weighted frequencies.
By combining these two summaries, they suggest a novel similarity score which exploits the underlying structure of the social network, and improves upon content-only measures.
Each incoming document is assigned to its closest cluster unless its similarity score is significantly lower than that of other documents, in which case, the oldest cluster is replaced with a new cluster containing  the new document.
A sketch-based approximation technique is used to maintain node statistics and to calculate structural similarity, greatly increasing the efficiency of their approach.
The rate of growth of each cluster is measured, and clusters which grow past a certain rate are said to be events. Unfortunately, evaluation of this approach is lacking in a number of areas.
Although they rigorously evaluate the effectiveness of their proposed similarity measure, an evaluation of the detection capability of this approach is lacking as they considers spam to be an event, and does not present how many false-positive events were detected.

\subsection{Locality Sensitive Hashing (LSH) using Random Hyperplanes}
Earlier we discussed how the basic TDT approach as described in Section \ref{background:sec:tdt} could not scale to extremely high volume streams (even with the aid of an inverted index), making it infeasible to use for event detection on Twitter.
\cite{Petrovic:2010:SFS:1857999.1858020} proposed a solution to this using a technique called Locality Sensitive Hashing (LSH).
LSH uses a special hash functions that produce the same hash for documents that are similar, but not necessarily completely identical, allowing for documents that are close in vector space to be placed into the same bucket.
Using a set of random hyperplanes and a family of hash functions proposed by \cite{Charikar2002}, the buckets are defined by the subspaces between hyperplanes, such that the probability of similar documents being placed into the same bucket of a hash table is proportional to their similarity.

By using multiple hash tables, each with independently chosen random hyperplanes, the probability of the true nearest neighbour colliding in at least one table can be increased to any desired probability (at the cost of additional computations).
The number of hash tables $L$ needed to give the desired probability of \emph{missing} a nearest neighbour $\delta$ can be computed as:
\begin{displaymath}
	L = log_{1-P^k_{coll}} \delta
\end{displaymath}
where $k$ is the number the number of hyperplanes, $P_{coll} = 1-\frac{\theta(x,y)}{\pi}$ where $\theta(x,y)$ is the expected angle between similar documents $x$ and $y$.
Petrivic et al. use values of $k = 13$, $\theta(x,y) = 0.2$ and $\delta = 0.025$ to compute $L$, values which we also use when replicating their work in Chapter \ref{chapter:collection}.

\subsubsection{Variance Reduction}
Although the use of LSH improves computational efficiency, it proves to be considerably less effective than the standard TDT approach.
This is because LSH is most effective when the true nearest neighbour is very close (i.e. has a high similarity) to the query document.
If the nearest neighbour is far from the query document, then LSH often fails to find it.
To overcome this, \cite{Petrovic:2010:SFS:1857999.1858020} use a `variance reduction' strategy when a nearest neighbour is not found using LSH.
In these instances, the approach falls back the traditional TDT model, and uses an inverted index to retrieve a list of documents to search for a nearest neighbour.
However, unlike the basic TDT model, a limit is placed on the number of documents searched, and only the most recent 2000 documents returned from the inverted index are compared.
This helps to improve the effectiveness, and brings it inline with that of the basic model, whilst still being considerably more efficient.

\subsubsection{Additional Efficiency Enhancements}
Since the number of buckets is limited, in a streaming scenario where the number of documents in unbounded, the number of documents in each bucket will continue to grow as new documents are processed.
This would require an unbounded amount of space and the number of comparisons made would also grow for each new document, eventually making even the LSH approach infeasible.
To overcome this, \cite{Petrovic:2010:SFS:1857999.1858020} limit the number of documents in a bucket to a fixed number based on the expected number of collisions, which can be computed as $n/2^k$ where $n$ is the total number of documents and $k$ is the number of hyperplanes used.
When a bucket is full, the oldest document in the bucket is removed from that bucket (and only that bucket) to make room for new documents.
Due to the use of multiple independent hash tables, the removal of the document from a bucket does not prevent it from being retrieved from other hash tables, although eventually all documents are removed from all hash tables.

To keep the number of comparisons per document fixed, similarity comparisons are performed for at most $3L$ documents (where $L$ is the number of hash tables).
The documents are selected by counting the number of times a document collides across the $L$ hash tables, and taking the top $3L$ documents with the most collisions.

\subsubsection{Noise Removal}
Although the use of LSH improves computational efficiency and allows the application of a TDT approaches on Twitter-scale data, it does not address the issue of noise and spam, which is found in excess on Twitter but is not present in the TDT datasets.

\cite{Petrovic:2010:SFS:1857999.1858020} investigated a number of ranking approaches for ranking clusters produced by their system.
They found that ranking clusters by the number unique users was more effective than the raw number of tweets.
They also found that the amount of information contained in the thread, measured using Shannon entropy \citep{Shannon:2001:MTC:584091.584093}, was a good indicator of quality.
Shannon entropy, $H$, is measured as:
\begin{displaymath}
	H = -\sum_i{\frac{n_i}{N} \log \frac{n_i}{N}}
\end{displaymath}
where $n_i$ is the number of times term $i$ appears in a cluster,	and $N = \sum_i{n_i}$ is the same of all term occurrences in the cluster.
Clusters with a low entropy, defined as $H < 3.5$ by Petrović et al., are moved to the bottom of the ranking -- effectively marking them as noise.

\subsection{Supervised Approaches to Event Detection}
\cite{Sakaki:2010:EST:1772690.1772777} were concerned with the detection of specific types of event, in particular, earthquakes and typhoons, with the aim of issuing alerts to those in the path of these disasters. Their approach, although simple, is very effective. They specify a set of predefined keywords which are associated with the type of events they are trying to detect (e.g. earthquake, shake, cyclone, typhoon, etc.,). They monitor an incoming stream of tweets for these keywords, and for each tweet found, they classify it using a Support Vector Machine (SVM) as either event-related or not. If they find enough event-related tweets in a short period of time, then their systems decides that the event is real, and issues alerts to those who could be affected. Despite the effectiveness of their approach, there are obvious drawbacks. Firstly, it can only detect specific types of event, and secondly, it requires training data for each of the types of event we want to detect.

\subsection{Natural Language Approaches}
\cite{Choudhury11extractingsemantic} examined how linguistic features and background knowledge could be used to detect specific types of sports events with high precision. However their approach requires significant domain knowledge of the sporting events and large amounts of manual labelling to prepare a classifier for each specific type of event, making it difficult and time-consuming to generalize.

\cite{Ritter:2012:ODE:2339530.2339704} used named entities, ``event phrases'' and temporal expressions to extract a calendar of significant events from Twitter.
However, their approach requires that tweets contain temporal resolution phrases, such as ``tomorrow'' or ``Wednesday'' for their approach to resolve between an entity and a specific time.
This means that smaller events, which do no generate much discussion before they happen, are unlikely to be detected. Additionally unexpected events, which are often the events which are of most interest, are unlikely to be detected as they will be discussed as they happen, and without any temporal resolution phrases.

Most similar to our work is that of \cite{Reuters2017}, who use a partitioning technique to efficiently cluster documents by splitting the tweet term space into groups they call `semantic categories', such as named entities, mentions, hashtags, nouns and verbs.
To ensure that only novel events are reported, they compare each new event to old events, filtering out any events that have already been reported.
They merge events by comparing the top 20\% of entity terms, based on term frequency, which reduces event fragmentation.
They evaluate their approach using the Events 2012 corpus we create in chapter \ref{chapter:collection}, finding that their approach outperforms the LSH approach when measured using NMI and B-Cubed cluster performance metrics.
Although this approach outperforms their baseline approach, the LSH approach of \cite{Petrovic:2010:SFS:1857999.1858020} using their chosen evaluation metrics, they do not report precision or recall values, making it difficult to compare directly.

\subsection{Burst and Trend Analysis}

\section{Differences between Event Detection and other IR Tasks}
Despite being viewed as an IR task, Event Detection has a number of characteristics that differ from more common IR tasks. The most obvious of these is the lack of a user-provided query.
Rather than searching for documents relevant to a specific query, event detection systems first must detect new events and track which documents belong to the same topic.
This means that the notion of relevance is much more ambiguous, and creates a number of issues with the evaluation of event detection approaches.